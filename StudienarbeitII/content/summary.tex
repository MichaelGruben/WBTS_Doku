\chapter{Ausblick}\label{ref:chaptSummary}
Dieses letzte Kapitel rundet die vorliegende Studienarbeit mit einem Blick in
die Zukunft ab. Es werden Ideen für weitere Versionen präsentiert interessante
Rückschlüsse für einen umfänglichen Umgang mit Masterly Mate vorgestellt.

\section{Ideen für weitere Versionen}\label{ref:weitereIdeen}
Es hat sich gezeigt, dass das in der Vorlesung Gamification entstandene Konzept
bei weitem nicht umgesetzt werden konnte. Allein die Realisierung der
grundlegenden Gedanken war in der gegebenen Zeit möglich. Daher folgt an dieser
Stelle eine Auflistung von Ideen für weitere Versionen von Masterly Mate.

\subsection{Für Anwender}
\begin{description}
\item[Liste absolvierter WBTs] Bisher gibt es für Lernende keine
direkte Möglichkeit auf absolvierte Tests in WBTs zurückzusehen. Mit einer
automatisch generierten Liste, die über die Navigation erreichbar ist, wird dies
möglich.
\item[Profilbild] Gewöhlich identifizieren sich die Nutzer einer Plattform
anhand ihres Avatars oder Profilbildes. In Masterly Mate muss diese Funktion
noch im Model für Nutzer hinzugefügt werden. Unter Umständen kann auch eine
Abfrage auf die bekannte Avatar-Plattform gravatar.com inkludiert werden.
\item[Jährliche Tests] Die tief im Konzept verwobenen jährlichen Tests fehlen in
der ersten Version von Masterly Mate noch. Dies ist auch darauf zurückzuführen,
dass bisher nur der Rahmen geschaffen wurde. Der Inhalt, wie beispielsweise
diverse WBTs, Themen und Tests fehlen für eine umfangreiche Erfahrung.
\item[Rangverlust für Tutoren] So fehlt im Zusammenhang mit dem
vorangegangenen Aspekt auch der Rangverlust für Tutoren bei einem Quartal
ohne gegebener Unterweisung.
\item[Quickstart] Neulinge haben es derzeit noch schwer. Es fehlt an
einem Quick-Start Guide, einer (Video-)Instruktion oder einem (Video-)Tutorial.
So wird der Einstieg erheblich vereinfacht.
\item[Gesamtranking] Mit einem Gesamtranking können sich
interessierte Lernende gegenüber Anderen vergleichen. Diese Übersicht
soll keinen Konkurrenzkampf auslösen und wurde daher bisher nicht
berücksichtigt. Die Art der Umsetzung erfordert eine ausgeklügelte
Konzeption.
\item[Open-ID] Um einen Login zu vereinfachen kann ein ID-Pool, wie
beispielsweise Open-ID abgefragt werden. So muss sich der Nutzer auf
Masterly Mate kein extra Konto einrichten. Mit einer ID, die er auf
diversen Plattformen nutzt kann er so auch auf Masterly Mate einfach
sein Passwort eingeben und ist ohne explizite Registrierung
eingeloggt.
\item[Newsletter] Ein persönlicher Newsletter kann an
interessierte Nutzer versendet werden. Dieser enthällt
persönliche Statistiken oder Informationen über neu
eingegangene oder geänderte WBTs. Die genaue Zusammenstellung kann im Rahmen
dessen Konzeption beschlossen werden.
\item[Realitätsnahe Tests] Mit Tests, die denen in Schulen oder Hochschulen
ähneln können sich Lernende in Prüfungssituationen üben. Bei eventuell
auftretenden können sie auf die bereits vermittelbaren Tutoren zurückgreifen. 
\end{description}

\subsection{Anwendungsintern}
\begin{description}
\item[Forum] In einem Forum können allgemeine Themen diskutiert werden. Hier
können neue Ideen für die Strukturierung von Masterly Mate entstehen oder über
alltägliches philosophiert werden.
\item[Mailerfunktionalität] Der Mailer ist ein sehr akutes Thema. Nutzer
erhalten so eine bestätigungs E-Mail für die Registrierung und andere relevante
Informationen.
\item[Refactorings] Für die bessere Umsetzung von DRY und KISS sollten
gelegentlich Refactorings angebracht werden. So wird die Programmcodequalität
verbessert und die Quelltexte werden einfacher lesbar und somit für eine
Community an Programmierern besser zugänglich.
\item[Tests] RoR bietet ein Unit-Test-Framework, welches aus Zeitgründen bisher
keine Verwendung fand. Tests sollten jedoch angebracht werden, um die
Korrektheit zu wahren und das YAGNI-Prinzip einfacher umzusetzen.
\item[Implementierung einer SCORM RTE und eines SCORM Players] Die vollständige
Implementierung von SCORM ist ein weiteres akutes Thema, welches in Angriff
genommen werden sollte. Analog der Arbeit aus \cite{mitter:2005} sollte eine
SCORM-RTE und je nach Bedarf ein SCORM-Player implementiert werden, der die
Kommunikation zwischen Masterly Mate und den WBTs gewährleistet und so die
Vergabe der Wertung automatisiert.
\end{description}

\subsection{Überarbeitungen für das Konzept}
\begin{k}
\begin{description}
\item[Investierte Zeit] Investierte Zeit mit in das Konzept für Belohnungen
einbringen
\item[Wertung für WBTs] Am Ende jedes WBT eine Wertung abgeben (zu schwer/zu leicht) -> tendiert
  eine Wertung zu stark in eine Richtung, wird das WBT dem am nächsten passenden
  Rang zugeordnet
\item[Adaptierung von Lerninhalten] Adaptierung von Lerninhalten, wie in
\cite{knall:2005}?
\item[Überraschungen] noch Überraschungen mit einbringen? \cite{korte:2009}.
Unlock big equipments for the selected design. Collections of Achievements
according to the current progression. Dynamic difficulty, i.e. riddles
\end{description}
\end{k}
\begin{k}

\subsection{Weitere Möglichkeiten für Gamification}
\begin{itemize}
\item Design Selection (Tamagochi, Avatar)
  \item Assistance possible
  \item Progress bar
  \item status message
  \item discussion room
  \item class room
  \end{itemize}
  
wie schaut das Gamification-Konzept von MM aus? Irgendwie ist Fortschrittsbalken
und Statistik und Möglichkeit als Tutor zu gering bis Meister. Gut ist der
Anreiz, ab Meister WBTs einbringen und editieren zu können. Wie war das mit dem
Avatar oder der UI, die immer besser gestaltet werden kann?

\end{k}

\section{Anschließende Rückschlüsse}\label{ref:anschlVorh}
\begin{k}
sehen, was wird
\begin{itemize}
  \item wirklich nur in Ausnahmefällen eine DE? (siehe
  \ref{ref:blendedLearning})
  \item (un)populär
  \item neue Verwendungszwecke
  \item nur eine Installation oder mehrere
\end{itemize}

kann mithilfe von Masterly Mate ein idealer Lehrer gefunden werden?
\end{k}