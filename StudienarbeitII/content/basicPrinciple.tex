% Meister, also Mitglieder mit bester didaktischer
% Qualifikation, werden Autoren von WBTs. Dadurch soll Qualität 

\chapter{Das zugrundeliegende Prinzip}
Zweck dieses und der folgenden Kapitel ist die Aufnahme der Grundlagen aus
Kapitel \ref{ref:basics} und deren Kopplung an den im Anschluss beschriebenen Aufbau der
Applikation.

\section{Namensgebung}\label{ref:naming}
Der Name "`Masterly Mate"' ist zusammengesetzt aus der Bezeichnung des höchsten
Rangs im Dreyfus-Modell (siehe Abschnitt \ref{ref:dreyfus}) und dem Namen eines
beliebten Getränks in Informatikerkreisen, beziehungsweise dem englischen
Begriff für Kumpel oder Kamerad.

So lässt sich der Name frei als meisterlicher Kamerad übersetzen, was die
erwünschte offene und freundliche Kommunikation auf der Plattform ausdrücken
soll.

\section{Freie Software}\label{ref:freeLicensesConcept}
Um Offenheit gleich bei der Entwicklung zu berücksichtigen, wird Masterly Mate
unter einer freien Lizenz, wie sie in Abschnitt \ref{ref:freeLicenses}
beschrieben ist, veröffentlicht werden. Damit kann jeder interessierte den
Quelltext einsehen und bei Bedarf selbst Hand anlegen, um Funktionalitäten zu
verbessern oder neue hinzuzufügen. So gibt es auch keine Falltüren im Sinne von
ungewünscht übermittelten und verwendeten Informationen und damit fehlender
Transparenz, wie beispielsweise bei Facebook oder Google.

Auch das vorliegende Dokument wird unter einer freien Lizenz, der \ac{GFDL} zur
Verfügung gestellt. So befindet sich nach der Eigenständigkeitserklärung ein
Lizenzhinweis. Nach den Vorgaben wird im Anhang ab Seite \pageref{ref:gfdl} die
komplette Lizenz aufgeführt. Daraus folgt, dass jeder Interessierte das Projekt
und dessen Ursprünge verfolgen kann. Auch bietet das Dokument Einblicke in Ideen
für weitere Versionen in Abschnitt \ref{ref:weitereIdeen} und anschließende
Vorhaben in Abschnitt \ref{ref:anschlVorh}.

\section{Die Idee}
Ziel des Systems ist die Vermittlung von Lerninhalten in einer sich gegenseitig
Unterstützenden Gemeinschaft. Zu diesem Zweck folgt das Konzept einer Art
Mischung aus Lern- und Datingplattform -- es werden Lerninhalte bereitgestellt,
zu denen Tutoren vermittelt werden.

Ein Anwender, der eine fachliche Herausforderung sucht oder sich in einem Fach
weiterbilden möchte, wird sich dem Bearbeiten von WBTs widmen. Mit Bestehen der
darin enthaltenen Quizes sammelt er Punkte für seinen fachlichen Rang. Unter
Umständen nimmer er dabei Hilfe von einem Tutor in Anspruch. Masterly Mate
bietet dazu eine regionale Suche an, mit deren Hilfe Lernende und Lehrende
aufeinander treffen. Insgesamt realisiert dies das Prinzip von Blended Learning
aus Abschnitt \ref{ref:blendedLearning}.

Ein Tutor erhält eine gute oder schlechte Bewertung. Damit verbessert oder
verschlechtert sich sein didaktischer Rang.