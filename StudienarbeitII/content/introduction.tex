\chapter{Einleitung}\label{ref:chaptIntroduction}
Basierend auf der Studienarbeit "`Analyse von Authorensystemen für ein WBT zu
Vorlesungszwecken von Michael Gruben \cite{gruben:2012} wird in dieser
Studienarbeit ein System aus \ac{WBT}s, ein \ac{LMS}, geschaffen. Das Konzept
für das Produkt des Projektes ist im Rahmen der Vorlesung "`Gamification"' entstanden.

Dabei handelt es sich grundsätzlich um eine Blended Learning Plattform, die
interessierten Lernenden eine zentrale Anlaufstelle bietet. Es werden also
eLearning und persönliches Lernen miteinander kombiniert. Umrahmt und
gamifiziert wird die Idee mithilfe des Dreyfus fünf Etappen Modells mentaler
Aktivitäten. Die in dieser Studienarbeit verwendeten Bezeichnungen unterliegen
gegebenenfalls weiteren Änderungen und sind für die deutschsprachige Version der
Plattform bestimmt.

Inhalte der vorliegenden Studienarbeit sind Einblicke in die Entwicklung des
ersten Prototyps. Zu diesem Zweck ist die Arbeit in vier Teile geteilt. Teil
\ref{ref:partPrep} enthällt Ausführungen über vorbereitende Maßnahmen. Der
Projektplanung und die Definition und Erklärung der benötigten Grundlagen für
das Verständnis der folgenden Kapitel. Anschließend zeigt Teil
\ref{ref:chaptConcept} die Konzeption und damit die grundlegende Idee und die
geplante Architektur. Darauf aufbauend wird in Teil \ref{ref:partImpl} näher auf
den tatsächlichen Entwurf und dessen Umsetzung eingegangen.
Im darin enthaltenen Kapitel \ref{ref:chaptScript} wird konkret auf Klassen und
Methoden eingegangen, welche die Realisierung bestimmter Use-Cases zum Ziel haben.
Kapitel \ref{ref:chaptImplementation} zeigt, wie der Entwurf letztlich
realisiert wird. 

Um die Arbeit abzuschließen folgen in Teil \ref{ref:partRefl}
zusammenfassende Worte. Es wird in Kapitel \ref{ref:chaptSummary} ein Ausblick
auf mögliche Funktionen weiterer Versionen gegeben. Kapitel
\ref{ref:chaptConclusion} schließt mit einem Fazit.
\newpage
Am Ende des Projekts steht ein funktionierender Prototyp, der die wesentlichen
Funktionen beherrscht. Weiterhin wird ein Konzept entwickelt worden sein,
welches das Projekt an zentralen Stellen bekannt macht und so für eine rege
Beteiligung sorgen soll. Mit der Namensgebung "`Masterly Mate"'\footnote{weitere
Details zur Namensgebung in Abschnitt \ref{ref:naming}} wurde bereits vor dem
eigentlichen Projektstart ein wesentlicher Schritt zur Bekanntmachung getan.