\chapter{Zusammenfassung der Ergebnisse}\label{ref:chaptConclusion}
Die in Kapitel \ref{ref:projektPlanung} vorgestellte Idee von Masterly Mate
wurde mit Abschluss der Studienarbeit grundlegend in einer Webapplikation
umgesetzt. Für den Zweck der Dokumentation folgten in Kapitel \ref{ref:basics}
die Erläuterung einiger Grundlagen, die im weiteren Verlauf für die beschriebene
Realisierung verwendet wurden. Nennenswert sind dabei insbesondere die
Definition von Lernen, Motivation, Blended Learning, dem Dreyfus-Modell und
Gamification, welche im Projekt eine besondere Rolle einnahmen und besonders
häufig aufgegriffen wurden.

Im zweiten großen Komplex wurde eine Konzeption erschaffen, in der die
Grundlagen oder Teile davon zu einer Applikation gebündelt wurden, die letztlich
ein LMS im Stil von Blended Learning bilden. Es wurde erläutert, worum es sich
bei Masterly Mate genau handelt und wie es aufgebaut sein soll. Dabei wurde das
Dreyfus-Modell so aufbereitet, dass es zur Idee aus dem vorangegangenen
Abschnitt stimmig ist. Darüber hinaus wurden weitere essentielle Themen
angesprochen, die einen einfachen Umgang und simple Erweiterungen des LMS
möglich machen sollen. Es wurde insbesondere genaueres Augenmerk auf
Gamification, sowie das Generieren von Motivation gelegt.

Mit dem weiteren Fortschritt der Arbeit nahm das Konzept in einem
Implementierungsteil an Form an. Es wurde ein konkreter Entwurf geschaffen, der
Klassen und Methoden zeigt. Hinzu kamen Erläuterungen, die die fertige Anwendung
aus Programmierer- und Nutzersicht beschreiben. Im Umsetzungsteil, dem Kapitel
\ref{ref:chaptImplementation} geschah der Übergang von der Theorie hin zur
Realisierung mit Ausführungen darüber, wie der Entwurf in Programmcode umgesetzt
wurde. Es hat sich gezeigt, dass \ac{RoR} sehr mächtig ist. Innerhalb kürzester
Zeit entstehen Webapplikationen mit umfangreichen Funktionen. Für Masterly Mate
reichten vorhandene gems aus und es musste kein proprietärer Ruby-Code
geschrieben werden. Sehr problematisch war hingegen der Umgang mit SCORM. Diese,
recht komplexe, Spezifikation konnte längst nicht in vollem Umfang implementiert
werden. Es war für die erste produktive Version angedacht, eine Kommunikation
zwischen \ac{LMS} und \ac{WBT} bereitzustellen. Nach Recherchen hat sich
gezeigt, dass dieses Vorhaben den Umfang eines eigenen Projektes beansprucht.
Alternativ ist die Verwendung des SCORM-Players, wie in \cite{mitter:2005} und
\cite{knall:2005} zu überdenken. Es steht auch zur Diskussion, ob das gewählte
scorm-gem die gewünschten Ziele in Kommunikation mit einer SCORM-RTE fehlerfrei
erfüllt. Derzeit entstehen, wie bereits in Abschnitt \ref{ref:implSCORM}
erwähnt, Fehler bei der Validierung von PIFs.\label{ref:problems}

Mit den aufgekommenen Herausforderungen ist mit Kapitel \ref{ref:chaptSummary}
ein Passus entstanden, in dem weitere Ideen für weitere Versionen beschrieben
sind. Dies zeigt, dass die Idee wesentlich mehr Potential beherbergt, als in
dieser Studienarbeit allein zeitlich realisiert werden konnte. Darüber hinaus
sind an dieser Stelle Thesen aufgestellt, die Ausgangspunkt für weitere
Untersuchungen sein könnten.

Mit der Konzeption einer Idee und der Umsetzung in eine Webapplikation wurde mit
Masterly Mate letztlich ein Modell vorgeschlagen, welches auf spielerische Weise
Lerninhalte nach dem Prinzip des Blended Learning vermittelt. Dazu wird ein
passender Tutor anhand des Dreyfus-Modells gewählt. Zusammenfassend wurden
demnach sehr moderne Lernmodelle mit klassischen in einer Lernplattform
verknüpft.