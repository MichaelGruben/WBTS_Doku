\chapter{Entwurf}\label{ref:chaptScript}
Der in diesem Kapitel beschriebene Entwurf zeigt konkret, wie das Konzept von
Masterly Mate umgesetzt werden wird. Hier werden Schemata und Architekturen
entwickelt, die im Kapitel \ref{ref:chaptImplementation} in Programmcode
umgesetzt werden.

\section{Realisierungsmethodik}
Nachdem in den vorangegangenen Kapiteln das Konzept von Masterly Mate erläutert
wurde, stellt sich die Frage nach einer geeigneten Möglichkeit zur Umsetzung. Da
die Anwendung stets verfügbar, leicht erreichbar, modular und einfach zu
verwalten sein soll, ist \ac{RoR} das Mittel der Wahl. Dieses bietet viele
interessante Features in Form von sogenannten gems, die dank einer regen
Community stets aktualisiert und erweitert werden. Zudem unterstützt es moderne
Programmierparadigmen, wie \ac{DRY} und \ac{KISS}. Dadurch bleibt die Anwendung
aus Sicht der Programmierer übersichtlich und erscheint sehr strukturiert. Das
das Framework der \ac{MVC}-Architektur folgt, schafft einen weiteren Grundstein
zur Trennung von Zuständigkeiten\footnote{bekannter unter "`separation of
concerns"'} und sorgt auch damit für Übersichtlichkeit. 

Weiterhin ist dieses Framework für die Weiterentwicklung im
OpenSource-Bereich prädistiniert, da damit bisher populäre
Webanwendungen, wie Twitter, realisiert wurden.

Darüber hinaus wird darauf geachtet, die Komponenten nach und nach nur dann zu
entwickeln, wenn sie tatsächlich gebraucht werden. Diese Vorgehensweise nach dem
\ac{YAGNI}-Prinzip beugt ein überlaufenes, unübersichtliches und schwer
zu wartendes Produkt vor.

\section{Internationalisierung}\label{ref:internationalisierung}
\begin{k}
Keine Internationalisierung in WBTs! (WBTs brauchen Attribut: Sprache)
Was ist die Default-Language? -> Sprache des Nutzers (User auch Attribut:
Muttersprache oder Herkunftsland)

Uebernimmt Julian?
\end{k}

\section{Beschreibung des Entwurfsklassendiagramms und
Use-Case}\label{ref:classModel}

\subsection{Use-Case Diagramm}
Im obigen Use-Case Diagramm, sind die einzelnen benötigten Komponenten
enthalten, wie sie die Analyse ergeben hat. Zunächst einmal lassen sich Aktoren
ausmachen, welche in System/Software-Aktoren und menschliche Aktoren aufteilen
lassen. 

Bei den Softwareseitigen Aktoren ergab die Analyse als Aktoren den Webserver,
ein Einzelnes WBT (sie sollen nach Abschluss die Punkte im System eintragen),
das Autorenwerkzeug und den Webbrowser. Auf der Seite der Aktoren ließe sich der
Lernende (Standard-User), Tutor, Autor und Admin ermitteln. Zu beachten ist,
dass die Aufteilung der menschlichen Aktoren auch eine Berechtigungshierarchie
darstellt, wobei ein Lernender die geringsten und ein Admin die meisten
Berechtigungen hat.

Die einzelnen Use Cases werden nun jeweils im folgenden kurz beschrieben. Zu
beachtenn ist dabei noch, dass diese Analyse sich nicht zwingend mit dem
Ergebnis aus der Studienarbeit deckt, da Funktionalitäten auf spätere Releases
verschoben worden sind.

\subsubsection{WBT durcharbeiten} 
Agierende Aktoren: \begin{itemize}
  \item WBT
  \item Lernender
  \item Tutor
  \item Autor
  \item Admin 
\end{itemize}

Ein Benutzer startet ein WBT und schließt es erfolgreich ab bzw.
scheitert oder beendet es vorzeitig. Im Anschluss daran, werden die erreichten
Punkte im System eingetragen. In der Version der Studienarbeit, wird dies noch
vom Benutzer selbst durchgeführt. Später soll dies das WBT übernehmen.
	
\subsubsection{Bewerten}
Agierende Aktoren: \begin{itemize}
  \item Lernender
  \item Tutor
\end{itemize} 

Den Use Case Bewerten gibt es in zwei Ausprägungen. Die erste Ausprägung ist die
Bewertung eines Tutors. Hat ein Lernender oder eine anderer Tutor (im folgenden
beide als "Lernende" bezeichnet) von dem Betreffenden Benutzer Hilfe erhalten,
so kann der Lernende im Anschluss daran diese Hilfe im System bewerten.
Dies kann er mit Sternen und wahlweise mit einem Kommentar tun (Kommentar noch
nicht in der Alpha-Version). Die Zweite Ausprägung ist die Bewertung eines WBTs.
Ein Benutzer muss das WBT dafür abgeschlossen haben. Ebenso wie beim Tutor kann
er dies über Sterne und Kommentare tun.
	
\subsubsection{Sprache verwalen}
Agierende Aktoren: \begin{itemize}
  \item Alle menschlichen Aktoren (Ausprägung Sprache
hinzufügen kann nur der Admin)
\end{itemize}

Ebenso wie im obigen Use Case gibt es hiervon zwei Ausprägungen.
Nummer Eins ist der Use Case Sprache wechseln. Dieser kann von jedem Benutzer
durchgeführt werden und bewirkt, dass alle Zeichenketten in MasterlyMate, welche
in der betreffenden Sprache vorhanden sind, ausgetauscht werden. Use Case Nummer
Zwei kann nur von einem Administrator durchgeführt werden. Und bewirkt die
Erzeugung einer weiteren Sprache, welche eingestellt werden kann. Gegenwärtig
wird dies noch direkt durch eine Änderung der Sprachdateiten getan.

\subsubsection{Profil verwalten}
Agierende Aktoren: \begin{itemize}
  \item Alle menschlichen Aktoren
\end{itemize}

Auch dieser Use Case verfügt über zwei Ausprägungen. Der erste Use Case ist das
bearbeiten eines Profils. Dies kann nur der besitzer des jeweiligen Profils.
Den anderen Use Case kann hingegen jeder Benutzer durchführen. Es handelt sich
dabei um das Betrachten eines Profils. Dies ist insbesondere für die
Kontaktaufnahme eines Benutzers zu einem Tutor notwendig.

\subsubsection{Suchen}
Agierende Aktoren: \begin{itemize}
  \item Alle menschlichen Aktoren
\end{itemize}

Es können sowohl Tutoren als auch WBTs gesucht werden. Dabei wird jeweils
berücksichtigt welches Themengebiet gefragt ist.
	
\subsubsection{Themengebiet abfragen}
Agierende Aktoren: \begin{itemize}
  \item Webserver
\end{itemize}

Der Webserver überprüft bei einer Suchanfrage die einzelnen Themen.
	
\subsubsection{Themen verwalten}
Agierende Aktoren: \begin{itemize}
  \item Admin
  \item Webserver
\end{itemize}

Nur der Admin kann die beiden Ausprägungen dieses Use Cases durchführen, welche
das Hinzufügen und Löschen eines Themas darstellen.
	
\subsubsection{WBT verwalten}
Agierende Aktoren: \begin{itemize}
  \item Admin
  \item Autor
  \item Webserver
\end{itemize}

Dieser Use Case besitzt drei Ausprägungen. Admin und Autor können WBTs
einbinden. Es kann jedoch nur der User, welcher ein WBT eingebunden hat, dieses
auch wieder entfernen bzw. ersetzen. Dies gilt nicht, wenn der betreffende
Benutzer ein Administrator ist. Dieser kann jedes WBT löschen oder ersetzen.
	

\subsection{Entwurfsklassendiagramm}
In dem Entwurfsklassendiagramm, sind die Models und ihre Beziehungen zueinander
dargestellt. Diese Modelle bilden einzelne Konzepte im System ab. So stellt User
ein Zentrales Modell dar, welches als Abbildung eines Benutzers im System für
die Authentifizierung verantwortlich trägt. Außerdem werden dem User
verschiedene Themen und WBTs zugeordnet. Diese Verbindungen benötigen jeweils
Assoziationsklassen. Dies liegt daran, dass es in dieser Verbindung Attribute
gibt, welche sich nicht eindeutig User oder WBT bzw. Thema zuordnen lassen. Bei
der Verbindung User-Thema benötigt man ein Attribut für die Punkte, welche
bisher von dem User in dem System erziehlt worden sind. Außerdem muss klar sein
welchen Rang ein User in dem jeweiligen Thema inne hat. Die Verbindung User-WBT
dagegen benötigt eine Variable um zu hinterlegen ob ein User das WBT bereits
einmal abgeschlossen hat und wenn ja, wie viele Punkte er erreicht hat. Das
WBT-Model verfügt über eine ID, sowie über einen Pfad zur eigentliche
SCORM-Datei. Wie bereits erwähnt ist es einem oder mehreren Themen (Topics)
zugeordnet.\label{ref:objectWBT}

\section{Funktionalitäten aus Nutzersicht}
Prinzipiell ist Masterly Mate aus zwei Komplexen aufgebaut. Zum einen kann sich
ein Nutzer fachlich weiterbilden. Zum Anderen bietet ein Nutzer als Tutor seine
Hilfe für ein bestimmtes Fachgebiet an.

\subsection{Masterly Mate für Lernende}
Aus Sicht der Lernenden baut sich Masterly Mate aus den folgenden Komponenten
auf.
\subsubsection{Registrieren und Einloggen}
Ein neuer Nutzer wird sich zunächst registrieren. Ist dies bereits geschehen,
kann er sich einloggen und sich dem Bearbeiten von WBTs oder der Administration
seines Profils widmen.

\subsubsection{Durcharbeiten von WBTs}
Ist ein Nutzer an Weiterbildungsangeboten interessiert, so geht er ein oder
mehrere WBTs durch. Dazu erhält er nach einer passenden Filterung der Ergebnisse
anhand einer Themenwahl und des dazugehörigen fachlichen Ranges (siehe Abschnitt
\ref{ref:autoResult}) eine verfügbare Liste an WBTs, die er nach seinem Gusto
bearbeiten kann.

\subsubsection{Tutorensuche}
Stößt der Lernende beim Durcharbeiten von WBTs auf ein fachliches Problem, so
kann er einen Tutor aufsuchen. Die Darbietung passender Tutoren erfolgt ebenso
nach einer automatischen Filterung. Die Filterung wird auch hier anhand des
fachlichen Rangs des Lernenden und des Themengebiets vorgenommen. Hinzu kommt
der didaktische Rang des Tutoren und die jeweils angegebene Postleitzahl
Lernenden und des Tutors, sodass ein Treffen aufgrund der räumlichen Nähe
einfacher möglicht wird.

\subsubsection{Profil administrieren}
Jedem Nutzer ist es erlaubt, sein eigenes Profil zu administrieren. Dort kann er
sein Profilbild und andere persönliche Angaben, wie Spitzname, Name und
Postleitzahl ändern. Zusätzlich kann er hier eine Option anhaken, die ihm zum
Tutor macht. Damit erscheint er für andere Lernende in den Suchergebnissen für
passende Tutoren. Demgegenüber kann er den Haken wieder entfernen, falls er kein
Tutor mehr sein möchte.

\subsubsection{Navigation, Impressum, Kontakt}
\begin{k}
Uebernimmt Julian
\end{k}

\subsubsection{Themen}
Themen fungieren für den Benutzer zum Einen als Suchfilter bei der Suche nach
den WBTs. Dies sorgt dafür, dass der Benutzer ohne umschweife auf WBTs zugreifen
kann, welche für ihn interessant sind. Außerdem dienen Themen dazu, die
unterschiedlichen Ränge, welche ein Benutzer zur selben Zeit in Masterly Mate
haben kann, voneinanderabzugrenzen. Jeder Benutzer kann in einem Thema nur genau
einen Rang inne haben. Umgekehrt kann er jedoch in beliebig vielen Themen einen
Rang haben.

\subsection{Masterly Mate aus Sicht eines Tutors}
Ein Tutor ist quasi eine erweiterte Form eines Lernenden. So bleiben dem Tutor
die verfügbaren Möglichkeiten eines Lernenden erhalten. Hinzu kommen zwei
weitere Funktionen.

\subsubsection{Lernende unterstützen}
Lernende werden gelegentlich an ihre Grenzen stoßen und Tutoren zu Rate ziehen.
Als Tutor auf Masterly Mate ist man, da man sich selbst in beliebigen
Fachgebieten weiterbilden kann, mit einem Lernenden nahezu gleichgestellt. Dem
Lernenden wir damit eine Hilfe gebende Person auf Augenhöhe vermittelt. 

Im Konzept von Masterly Mate ist nur eine Hilfe der beschriebenen Art
berücksichtigt. Tutoren können sich jedoch darüber hinaus auch auf eine
beliebige andere Art an Lernende wenden, wie zum Beispiel das geben von
Workshops oder halten von Präsentationen.

\subsubsection{WBTs verbessern und hinzufügen}
Hat ein Tutor den Rang des Meisters erreicht, so ist es ihm aufgrund seiner
herausragenden didaktischen Leistungen erlaubt, bestehende WBTs zu verbessern
oder neue zu entwickeln und auf die Plattform zu laden.