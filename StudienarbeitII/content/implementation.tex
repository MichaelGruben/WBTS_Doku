\chapter{Umsetzung}\label{ref:chaptImplementation}

\section{Nutzerverwaltung}\label{ref:sectNutzerverwaltung}
\begin{k}
Im Rahmen dieser Studienarbeit besitzt Masterly Mate neben der Zugangskontrolle
einen eingebauten Mechanismus für die Zugriffskontrolle. Die registrierten
Benutzer werden in der Datenbanktabelle users festgehalten. Diese besitzen neben
dem Attribut Passwort und Benutzername weitere essentielle Attribute wie z.B.
Geburtsdatum, Vorname, Nachname und Geschlecht. Näheres zur Implementierung der
Zugangskontrolle wird im Kapitel \ref{ref:sectAuthentifizierung} erläutert.
Für den Zugriff auf vertrauliche Ressourcen, wie z.B. das Benutzerprofil, die
vom Benutzer durchgeführten Assessments und Examen sowie die zugänglichen WBTs
und die Nutzerstatistik, müssen vor unberechtigten Nutzern geschützt werden.
Für diese Autorisierungsvorgabe wurde das Gem mit dem Namen CanCan
\footnote{https://rubygems.org/gems/cancan} eingesetzt. Dieses Gem ermöglicht
die zentrale Zugriffsverwaltung auf beliebige Ressourcen. Die Klasse Ability des
Gems ist dabei die einzige Klasse die für die Autorisierung von wichtiger
Bedeutung ist. Diese befindet sich nach der Installation von CanCan im
Verzeichnis app/models/. Die Autorisierung auf Ressourcen wird mit Hilfe der
Methoden can? und cannot? realisiert. Diese Methoden erwarten als ersten
Parameter ein vordefiniertes Symbol, welches die Zugriffsart festlegt. Häufige
Zugriffsarten sind z.B. :manage, :destroy, :update, etc. Im Grunde wird hier der Name der
Aktion angegeben, der auf diese Ressource möglich bzw. nicht möglich ist. Bei
dem zweiten Parameter der Methoden can? und cannot? handelt es sich entweder um
den Namen einer Ressource oder einer Instanz einer Ressource. Nutzt man den
Namen einer Ressource, dann bezieht sich die definierte Aktion auf sämtliche
Instanzen dieser Ressource. Im anderen Fall bezieht sich die Aktion nur auf
diese eine Instanz. Die Autorisierung erfolgt in Masterly Mate direkt im
Konstruktor der Ability Klasse. In Masterly Mate wird eine Gruppenbasierte
Rechtevergabe durchgeführt. Mithilfe der Methode group?, welche in der Klasse
des User Models definiert ist, kann man unter Angabe des Namens einer Gruppe,
überprüfen, ob dieser Nutzer der entsprechenden Gruppe zugehört oder nicht.
der Konstruktor der Klasse Ability erwartet eine Instanz von der Klasse
User. Wird dem Konstruktor eine gültige User Instanz übergeben, so wird
mit dieser Instanz weiter operiert. Andernfalls wird eine neue Instanz der
Klasse User erzeugt und als Gastnutzer interpretiert. Dieser würde lediglich
Zugriff auf öffentliche Ressourcen besitzen und könnte sich erst
Authentifizieren, wenn dieser sich registriert hat. Im anderen Fall wird
überprüft ob der aktuelle Nutzer Mitglied der vordefinierten Gruppe
Administrator ist. Wenn dem so ist, dann besitzt der Nutzer sämtliche Rechte auf
alle Ressourcen. Der einzige Vorgang, der auch einem Administrator verwehrt
wird, ist das Entfernen des eigenen Kontos, da ansonsten die Gefahr
besteht, dass kein Administrator in Masterly Mate zur Verfügung steht und das
Entfernen von bestehenden Gruppen. Ein Nutzer der Mitglied der Gruppe Registered
ist, hat die Möglichkeit das eigene Profil und dessen Assessments zu verwalten.
Dabei kann dieser sämtliche Operationen, bis auf das Entfernen des eigenen Profils,
durchführen. Weiterhin besitzen diese Nutzer lediglich einen lesenden Zugriff
auf Themen und einen ausführenden Zugriff auf WBTs.
\end{k}

\section{Lokationen}
Die Angabe von Lokationen bzw. Routen sind für REST-basierte Webanwendungen
unumgänglich. Diese Lokationen werden in RoR in der Datei 
config/routes.rb spezifiziert. Im Kapitel ref{ref:internationalisierung}
wurde erwähnt, wie I18n in Masterly Mate eingesetzt wird. In diesem Kapitel wird der
Zugriff auf eine Sprachdatei durch Lokationen erläutert. In REST-basierten
Webanwendungen gibt es verschiedene Möglichkeiten die gewünschte Sprache über
die URL anzugeben. Dies kann bspw. über einen Parameter in der URL oder
aber gleich unmittelbar vor dem Wurzelverzeichnis in der URL erfolgen. In
Masterly Mate wird letzteres bevorzugt. Für das Setzen des Ländercodes in der
URL wird dem Nutzer ein sogenannter Languge-Switcher angeboten. Dieses wird
durch eine Combobox repräsentiert, mit dessen Hilfe sich der Nutzer eine Sprache
aus der Liste aussuchen kann. Nachdem der Nutzer sich eine Sprache ausgesucht
hat, wird das onChange Event des DOM-Combobox-Elementes ausgelöst und mit Hilfe
eines reguären Ausdrucks der vorhandene Ländercode in der URL durch den
Ländercode der gewählten Sprache ausgetauscht. Die Klasse I18n besitzt das
Attribut locale. Dieses Attribut beinhaltet als Wert den Ländercode der zu
verwendenden Sprache. In der privaten Methode locale_path des Application
Controllers, wird dieses statische attribute gesetzt und die aktuelle URL
entsprechend angepasst. Die Optik des Language Switchers wird in dem partiellen
Layout unter app/views/shared/_language_switcher.html.erb festgelegt und kann
somit in jedem anderen View ohne Probleme eingebunden werden.
\begin{k}

\end{k}

\section{Suche}
\begin{k}
Uebernimmt Julian/Benni?
\end{k}

\section{SCORM}\label{ref:implSCORM}
Die SCORM-Funktionalität wird in dem Objekt "`wbt"' realisiert, welches in
Abschnitt \ref{ref:objectWBT} beschrieben wurde. 

Die darin enthaltene Upload-Methode, die beim Einfügen und Ändern eines WBTs
aufgerufen wird, speichert das WBT auf dem lokalen Speicher des Servers. Dabei
wird das \ac{PIF} mithilfe der Funktionen aus einem
scorm-gem\footnote{Ressource: \url{https://rubygems.org/gems/scorm}} direkt
entpackt. Dabei ist mit der Berücksichtigung der Validierung stets ein Fehler
aufgetreten. Für die erste Version wird daher keine Validierung unterstützt, das
PIF wird ohne Prüfung entpackt (siehe Abschnitt \ref{ref:problems}). Die
Fehlerursache liegt unter Umständen an dem Alter des gems. Eine genauere Betrachtung ist für
spätere Versionen von Masterly Mate angedacht. 

Mit dem Hochladen und Entpacken werden die nötigen Attibute für den Start des
WBT gepflegt. Dies ist zum einen der Paketname selbst und zum anderen der Pfad
zur Start-Datei des Root SCO. In einer start-Methode werden diese Attribute
ausgelesen und das WBT wird in einem neuen Fenster geöffnet. So kann das WBT den
Raum einnehmen, den es braucht. Masterly Mate bleibt damit unabhängig vom Stil
des Autorenwerkzeugs. Für die erste produktive Version fehlt es noch an einem
geeigneten RTE, da dessen Implementierung den Rahmen der vorliegenden Arbeit
sprengen würde. Daher erfolgt die Registrierung des Ergebnisses zunächst durch
eine manuelle Eingabe des Nutzers.

\section{Lizensierung}
Nach den in Abschnitt \ref{ref:freeLicenses} beschriebenen freien Lizenzen und
dem Einbringen der Idee in das Konzept (siehe Abschnitt
\ref{ref:freeLicensesConcept}) wurde für die Zwecke von Masterly Mate auf die
\ac{AGPL} zurückgegriffen. Da diese mit der \ac{GPL} kompatibel ist, wird
eine eventuelle Verbreitung der Software im Sinne von OpenSource möglich.
Darüber hinaus kann das Projekt unter anderen kompatiblen Lizenten verbreitet
werden \cite{fsf:2007}.

Mit der Nutzung der AGPL entsteht die Pflicht, den Quelltext der Anwendung
direkt als Download anzubieten. Dazu wird im Interface Masterly Mate
ein Link auf die GIT-Ressource im Footer angeboten. Zusätzlich wurde, wie bei
allen Lizenzen nötig, in jeder Datei ein Lizenztext vorrangestellt.

\section{Dokumentation}
Wie in Abschnitt \ref{ref:archDoc} beschrieben, wurde mithilfe der Befehlszeile
\textit{rake doc:app} eine Dokumentation der API erstellt und im Footer
mit einem relativen Link auf \textit{doc/app/index.html} referenziert. Die im
Projekt verwendete Version 10.0.2 von rake bedient sich RDoc in der Version
2.12.2 und dem Darkfish Rdoc Generator 3.

Zusammen mit den Anmerkungen im Quelltext entsteht so eine ausführliche
Dokumentation der Fähigkeiten und Schnittstellen. In einem Index über die
Klassen und Module kann gezielt nach bestimmten Funktionsweisen gesucht werden.
Die dazu angebrachten Kurzbeschreibungen und Verweise bieten einen einfachen
Überblick und verhelfen einem interessierten Anwender oder Programmierer zur
Transparenz über die Funktionsweise von Masterly Mate.

Die Dokumentation der API ist in der englischen Sprache gehalten, da diese für
gewöhnlich nicht multilingual geführt wird und Englisch als Sprache für
computerrelevante Themen anerkannt ist.

\section{Gestaltung}
Das Design von Masterly Mate ist in der ersten Version zunächst einmal von
Funktionalität geprägt. Jedoch finden schon einige der weiter oben aufgezählten
gestalterischen Grundkonzepte hier Eingang. So ist es der Erwartungskonformität
geschuldet, dass Navigation und Anzeige sich in optisch voneinander getrennten
Bereichen befinden. Ebenfalls von vielen anderen Seiten bekannt, ist dass
Konzept, die Änderung der Sprach oben rechts und damit abseits von allen anderen
Kontrollen zu platzieren. Die Navigation ist von der Anzeige durch das Gesetz
der Nähe und einen Sichtbaren Trennstrich getrennt. Um eine bessere
Übersichtlichkeit auf den WBTs zu gewährleisten, werden diese Außerdem in einem
neuen Tab geöffnet. Da Masterly Mate als gamifizierte Anwendung nicht vollkommen
auf ein Mindestmaß an ansprechender Optik verzichten kann, ist es in einem
warmen Orange gehalten. Die wenigen anderen Farben sind so gewählt, dass sie
nicht zu sehr in Kontrast zur Hauptfarbe stehen.

\section{Authentifizierung}\label{ref:sectAuthentifizierung}
\begin{k}
Im Kapitel ref{ref:sectNutzerverwaltung} wurde die Umsetzung der Autorisierung
in Masterly Mate erläutert. In diesem Kapitel wird nun die Umsetzung der
Authentifizierung kurz beschrieben. Damit sich ein Nutzer erfolgreich bei
Masterly Mate registrieren und im Anschluss darauf anmelden kann, wird ein
Authentifizierungsmechanismus benötigt, welches sicher und zudem eine geringe
Wartungskomplexität aufweist. Ein Gem, dass diese Voraussetzungen erfüllt, ist
das Ruby eigene Gem mit dem Namen bcrypt-ruby
\footnote{http://bcrypt-ruby.rubyforge.org/}. Dieses Gem nutzt für die
Passwortverschlüsselung einen SHA-Hash und kann daher für den
heutigen Stand der Technik als sicher eingestuft werden. Dieses Gem setzt
außerdem ein User Model voraus, welches zwei String-Attribute besitzt. Eines mit
dem Namen password_digest und eines mit dem Namen username. Die Methode
has_secure_password bildet den Authentifizierungsvorgang auf dieses Model ab und
muss daher in der Klasse User zu Beginn aufgerufen werden. Die Abbildung auf
dieses Model erfolgt durch Hinzugabe einer Methode mit dem Namen authenticate zu
dieser User Klasse. Der Registrierungsvorgang wird von dem User Controller
verwaltet. Der Authentifizierungsvorgang hingegen vom Session Controller. Der
Session Controller ist für die Verwaltung sämtlicher Sitzungen verschiedener Nutzer
zuständig. In der create Action des Session Controllers wird zunächst geprüft,
ob der anfragende Nutzer alle relevanten Informationen zur Authentifizierung
angegeben hat und ob diese Informationen korrekt sind. Diese beiden Vorgänge
werden von der Methode authenticate, welche auf der gefundenen User
Instanz ausgeführt wird, durchgeführt. Als Parameter erwartet diese Methode lediglich
das vom Passwort. War die Authentifizierung erfolgreich, wird der Nutzer auf die
Startseite weitergeleitet und die Nutzer-ID in dem
Sitzungs-Hash zwischengespeichert. Die destroy action wird aufgerufen, sobald
der Nutzer sich abmeldet. Dabei wird einfach die im Sitzungs-Hash gespeicherte
Nutzer-ID auf nil \footnote{äquivalent zu NULL} gesetzt. Der Nutzer wird danach
ebenfalls auf die Startseite weitergeleitet.
\end{k}

\section{Themen}
Der Model Topic, welches die Themen abbildet verfügt über die Attribute name und
parent\_name. Zweiteres dient zur Identifizierung des Überthemas. Die Wahl
dieses Form der Referenz zu verwenden ist der Datenbank geschuldet. Auf diese Weise
liegt der Fremdschlüssel jeweils auf der n-Seite der 1:n Beziehung.